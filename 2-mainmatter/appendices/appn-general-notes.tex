\chapter{General Notes}
For any package, always consult the official package manual and/or reference
for specific instructions on how to load them and the range of options
available. If you run into strange issues, feel free to ask for send
me an email or ask on the PhD or MS mailing lists.

{Note:} As is the case for many other programming languages, there are
lot of {\LaTeX} code snippets floating around at various forums on the
internet. While using these code snippets may (or may not) alleviate issues,
it is highly plausible that such code can have undesirable side effects, just
from being placed in a document. Further, a single code snippet can, in general,
have different effects depending on where it is placed in the document.
Therefore, the user is urged to inspect an official manual and/or reference
before inserting arbitrary code into the document. The code of
{iitmdissertation} largely conforms to the best practices and strictly
abides by the documentation of the packages that it loads.

\section{Source files} \label{sec:source-files}
The user must avoid editing the source files as far as possible. Commands do
not always do what one expects of them unless of course they are invoked in the
right context (and/or situation). {\ttfamily iitmdissertation} has been extensively
tested in many scenarios. However, it is very much possible that bugs exist.
Most errors in compilation arise from incorrect use of packages and/or their
facilities. Take care to ensure that you follow best practices while using
external packages. If you encounter an insurmountable issue despite your best
efforts, you are most welcome to contact the author for help.

Ensure that these files are not tampered with at any cost. The accompanying
template will not compile (and produce a PDF) if even one of these is missing. 
These are as follows: {\ttfamily iitm.bst}, {\ttfamily iitmdissertation.cls}, 
{\ttfamily iitmdissertation.sty} and the folder {\ttfamily zz-imp}.

\section{Some useful packages}
Quite a few packages are loaded by {iitmdissertation} which are very
commonplace in a standard document. In the event that certain customization of
one or more of the facilities offered by these packages is required, the user
is urged to look at the respective package's documentation.

\begin{table}
\captionabove{A list of commonly required (some of which are pre-loaded) packages along with links to
their documentation}
\centering
\begin{tabular}{>{\ttfamily}l p{0.7\textwidth}}
\toprule
{\rmfamily Package(s)} & Note(s) \\
\midrule
{enumitem}
    &
handles all itemized lists and descriptions
[\cite{enumitem}] \\

{booktabs} \& {array}
    &
handle all facilities related to tables [\cite{booktabs,array}]
\newline
{Caution: } Although, there are many table-related packages for {\LaTeX},
use of {booktabs} and {array} is the recommended way to handle
tabulation \\

{xcolor}
    &
handles all color related facilities. This package is loaded with the
{usenames,svgnames} option for a total of 170 color options. Consult page
43 of the {xcolor} manual for these color names [\cite{xcolor}]
\newline
{Caution: } Trying to load {xcolor} with other options might cause
issues (i.e, option clashes in {\LaTeX} parlance). If you really must have new
color(s) you can define it using the {definecolor} command. \\

{subcaption}
    &
This is the recommended package for handling sub-figures and
sub-captions. This is not loaded by default but is highly recommended if the
need for multiple figures with sub-captions arises [\cite{subcaption}] \\

{multirow}
    &
This is the recommended package for handling tabular-cells spanning multiple
rows. This is not loaded by default, since it is not
essential.[\cite{multirow}] \\

{glossaries}
    &
This is the default package used to handle glossaries and the abbreviations in
the prematter. {\ttfamily iitmdissertation} provides defaults for a simple glossary
customized to match the official guidelines of the institute as can be seen in
the template. [\cite{glossaries}] \\

{nomencl}
    &
This is the default package used to handle the notation chapter in the
prematter. {\ttfamily iitmdissertation} provides defaults for a simple notations
chapter conforming to the guidelines. [\cite{nomencl}] \\
\bottomrule
\end{tabular}
\end{table}

We hope this helps users to produce nicely formatted theses and simplifies his/her 
writing experience. Happy writing!